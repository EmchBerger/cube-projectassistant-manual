% \documentclass{article}

\documentclass[twocolumn]{article}


\usepackage{tcolorbox}

%%% \usepackage{tikz}
%%% % \pagecolor{olive!50!yellow!50!white}

\usepackage{xcolor}
\definecolor{chamois}{rgb}{1,.984314,.956863}  % RGB 255 247 231
\pagecolor{chamois}

\usepackage{geometry}
\geometry{verbose,a4paper,tmargin=10mm,bmargin=10mm,lmargin=10mm,rmargin=10mm}


\tcbuselibrary{skins}

\colorlet{xlightblue}{blue!5}

\newtcolorbox{beamerlikethm}[1]{
  title=#1,
  beamer,
  colback=xlightblue,
  colframe=blue!30,
  fonttitle=\bfseries,
  left=1mm,
  right=1mm,
  top=1mm,
  bottom=1mm,
  middle=1mm
}



\begin{document}


%%% \begin{tikzpicture}[remember picture, overlay]
%%% \shade[left color=red!50,
%%% right color=green!50
%%% ] (current page.north west) rectangle (current page.south east);
%%% \end{tikzpicture}



\twocolumn[
% \small % kleinere Schriftgröße
% \itshape % alles kursiv
%\blindtext % Blindtext
Das Sitzungswesen im CUBE PA ermöglicht Ihnen folgende Tätigkeiten: A: Eine Sitzungseinladung zusammenstellen und versenden B: Das Protokoll der Sitzung führen und versenden C: Pendenzen erstellen und nachführen D: Entscheide dokumentieren E: Sitzungseinladungen, Sitzungsprotokolle, Pendenzen und Entscheide suchen und lesen und bearbeiten
\vspace*{1.5cm} % manuell gesetzter, vertikaler Abstand
]

\noindent  % Zurücksetzen auf zwei Spalten



%%%%%%


\begin{beamerlikethm}{Sitzungswesen}
\begin{itemize}
  \item[$\Longrightarrow$] Wählen Sie aus dem Menü links 'Sitzungswesen' und 'Sitzungen'.
  \item[$\Longrightarrow$] Klicken Sie auf das 'Plussymbol' oben.
  \item[$\Longrightarrow$] Wählen Sie einen Sitzungstyp aus und geben Sie die Eckdaten zur Sitzung ein.
	\item[$\Longrightarrow$] Wählen Sie Teilnehmende aus, entweder einzeln oder mittels vordefinierter Teilnehmerliste.
  \item[$\Longrightarrow$] Speicher Sie diese Daten mit 'Übernehmen' ab. Nun erscheinen neue Eingabefelder.
  \item[$\Longrightarrow$] Sie können nun Traktanden erfassen (einzeln oder mittels Auswählen einer vordefinierten Liste).
  \item[$\Longrightarrow$] Fügen Sie bei Bedarf Beilagen / Dokumenten hinzu.
  \item[$\Longrightarrow$] Schliessend Sie den Vorgang erneut mit Klick auf 'Übernahme' ab.
  \item[$\Longrightarrow$] Kehren Sie zurück auf die Übersicht.
\end{itemize}
\end{beamerlikethm}



\begin{beamerlikethm}{Eine Sitzungseinladung bearbeiten}
\begin{itemize}
  \item[$\Longrightarrow$] Wählen Sie die gewünschte Sitzung aus. Klicken Sie auf den Titel. Die Optionen öffnen sich.
  \item[$\Longrightarrow$] Klicken Sie auf das Bearbeitungssymbol. Nun können Sie die gewünschten Änderungen vornehmen.
  \item[$\Longrightarrow$] Schliessend Sie den Vorgang erneut mit Klick auf 'Übernahme' ab.
\end{itemize}
\end{beamerlikethm}



%%%% andere Form

\begin{tcolorbox}[colback=blue!5,colframe=blue!40!black,title=Einladung versenden]
\begin{itemize}
  \item[$\Longrightarrow$] Kehren Sie zur Übersicht zurück
  \item[$\Longrightarrow$] Wählen Sie die gewünschte Sitzung aus und klicken Sie auf den blauen Titel. Die Optionen öffnen sich
  \item[$\Longrightarrow$] Klicken Sie auf das Briefsymbol (PDF Einladung versenden). Nun wird die Einladung erstellt.
  \item[$\Longrightarrow$] Sie können das erstellte PDF per Email oder auch per Post versenden.
\end{itemize}
\end{tcolorbox}



\begin{tcolorbox}[colback=blue!5,colframe=blue!40!black,title=Protokoll bearbeiten]
\begin{itemize}
  \item[$\Longrightarrow$] Kehren Sie zur Übersicht zurück
  \item[$\Longrightarrow$] Wählen Sie die gewünschte Sitzung aus und klicken Sie auf den blauen Titel. Die Optionen öffnen sich
  \item[$\Longrightarrow$] Klicken Sie auf das Protokollsymbol. Das Sitzungsprotokoll wird geöffnet.
  \item[$\Longrightarrow$] Sie können das erstellte PDF per Email oder auch per Post versenden.
\end{itemize}
\end{tcolorbox}



  %%%%%%
	% Spalte 2
	%%%%%%
	
	\begin{beamerlikethm}{Sitzungswesen}
\begin{itemize}
  \item[$\Longrightarrow$] Wählen Sie aus dem Menü links 'Sitzungswesen' und 'Sitzungen'.
  \item[$\Longrightarrow$] Klicken Sie auf das 'Plussymbol' oben.
  \item[$\Longrightarrow$] Wählen Sie einen Sitzungstyp aus und geben Sie die Eckdaten zur Sitzung ein.
	\item[$\Longrightarrow$] Wählen Sie Teilnehmende aus, entweder einzeln oder mittels vordefinierter Teilnehmerliste.
  \item[$\Longrightarrow$] Speicher Sie diese Daten mit 'Übernehmen' ab. Nun erscheinen neue Eingabefelder.
  \item[$\Longrightarrow$] Sie können nun Traktanden erfassen (einzeln oder mittels Auswählen einer vordefinierten Liste).
  \item[$\Longrightarrow$] Fügen Sie bei Bedarf Beilagen / Dokumenten hinzu.
  \item[$\Longrightarrow$] Schliessend Sie den Vorgang erneut mit Klick auf 'Übernahme' ab.
  \item[$\Longrightarrow$] Kehren Sie zurück auf die Übersicht.
\end{itemize}
\end{beamerlikethm}



\begin{beamerlikethm}{Eine Sitzungseinladung bearbeiten}
\begin{itemize}
  \item[$\Longrightarrow$] Wählen Sie die gewünschte Sitzung aus. Klicken Sie auf den Titel. Die Optionen öffnen sich.
  \item[$\Longrightarrow$] Klicken Sie auf das Bearbeitungssymbol. Nun können Sie die gewünschten Änderungen vornehmen.
  \item[$\Longrightarrow$] Schliessend Sie den Vorgang erneut mit Klick auf 'Übernahme' ab.
\end{itemize}
\end{beamerlikethm}



%%%% andere Form

\begin{tcolorbox}[colback=blue!5,colframe=blue!40!black,title=Einladung versenden]
\begin{itemize}
  \item[$\Longrightarrow$] Kehren Sie zur Übersicht zurück
  \item[$\Longrightarrow$] Wählen Sie die gewünschte Sitzung aus und klicken Sie auf den blauen Titel. Die Optionen öffnen sich
  \item[$\Longrightarrow$] Klicken Sie auf das Briefsymbol (PDF Einladung versenden). Nun wird die Einladung erstellt.
  \item[$\Longrightarrow$] Sie können das erstellte PDF per Email oder auch per Post versenden.
\end{itemize}
\end{tcolorbox}



\begin{tcolorbox}[colback=blue!5,colframe=blue!40!black,title=Protokoll bearbeiten]
\begin{itemize}
  \item[$\Longrightarrow$] Kehren Sie zur Übersicht zurück
  \item[$\Longrightarrow$] Wählen Sie die gewünschte Sitzung aus und klicken Sie auf den blauen Titel. Die Optionen öffnen sich
  \item[$\Longrightarrow$] Klicken Sie auf das Protokollsymbol. Das Sitzungsprotokoll wird geöffnet.
  \item[$\Longrightarrow$] Sie können das erstellte PDF per Email oder auch per Post versenden.
\end{itemize}
\end{tcolorbox}






\end{document}