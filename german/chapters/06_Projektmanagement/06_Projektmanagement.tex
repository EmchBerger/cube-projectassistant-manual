
\clearpage

\section{Anforderungs - und Mängelmanagement}

In diesem Kapitel wird das CUBE Anforderungs - und Mängelmanagement beschrieben.

\vspace{\baselineskip}

Das Modul 'Anforderungs - und Mängelmanagement' ist ein Hilfsmittel zur Erfassung und Verwaltung von technischen, baulichen, betrieblichen und organisatorischen Anforderungen auf allen Projektstufen. Der Workflow orientiert sich dabei an einem Statussystem, welches den Benutzer durch sämtliche Arbeitsschritte / Prozesse wie Anforderungen erfassen und validieren, Mängel aufnehmen und deren Behebung nachweisen u.a. führt. Mittels dem Modul 'Projektmanagement' haben Sie jederzeit den Überblick über eine Abnahme und wissen genau, welche Arbeiten noch erledigt oder nachgebessert werden müssen.

\vspace{\baselineskip}

Das Modul beinhaltet folgende 4 Kategorien:

\vspace{\baselineskip}

\begin{compactitem}
	\item Anforderungen
	\item Arbeitspakete
	\item Mängel
	\item Pendenzen
\end{compactitem}

\vspace{\baselineskip}

\textbf{Anforderungen:} Zuerst werden die Anforderungen erfasst, welche im Zusammenhang mit dem Projekt, den Vorbereitungen oder Umsystemen stehen.

\textbf{Arbeitspakete:} Sie beschreiben Arbeitsschritte eines Projekts oder fassen gewisse Arbeiten zusammen, welche bspw. einen Auftragsnehmer betreffen.

\textbf{Mängel:} Bezeichnen die Mängel/Fehler an Projektarbeiten, welche noch einer Korrektur oder Besserung bedürfen. Sie können mit Pendenzen oder Arbeitspaketen verknüpft werden. Die Mängel beziehen sich jeweils auf eine Anforderung. Mängel erhalten einen Status, welcher den Projektzustand wiedergibt (Beispielsweise: Freigegeben und Abgeschlossen). Das Statussystem kann kundenspezifisch konfiguriert werden.

\textbf{Pendenzen:} Die Pendenzen entsprechen den Pendenzen des Sitzungswesens und können Bestandteil eines Mangels sein. 

\vspace{\baselineskip}

\subsection{Benutzerrollen:} 
Benutzer können grundsätzlich in drei Rollen eingeteilt werden: Ersteller (Autor An-forderung), Validierer und Teilprojektleiter Validierung (TPL Validierung). Diese Benutzer-rollen haben zentrale Aufgaben im RAMS- und Validierungsprozess.

\subsubsection{Ersteller (Autor Anforderung):} 
Der Ersteller verfasst und bearbeitet Anforderungen ---ev Verweiss auf Kapitel Anforderungen erfassen (Kapitel xy)---. Er prüft die Anforderungen auf Vollständigkeit und schaltet den Status auf «Anforderungen erfasst» weiter. Wird der Status einer Anforderung zurückgesetzt, kann er diese erneut bearbeiten, splitten oder gruppieren. 
Ausserhalb der Bearbeitung von Anforderung hat der Ersteller nur eine Leseberechtigung. Jeder Ersteller kann jede Anforderung bearbeiten. Dies ermöglicht eine grosse Flexibilität beim Wechsel des Erstellers. CUBE AFM verhindert eine gleichzeitige Bearbeitung derselben Anforderung durch zwei Ersteller. Der zweite Ersteller wird darüber informiert, wer momentan den Anforderungseintrag bearbeitet. 

\subsubsection{Validierer:} 
Nach dem Statuswechsel von 'Erfassung abgeschlossen' zu 'Nachweisplanung' führt der Validierer eine Planung der Nachweisführung durch (Kapitel xy). Dabei erfasst er einen Eintrag zur Nachweisplanung in der Rubrik 'Validierung der Anforderung (Planung)', welche aufzeigt, wie er die Anforderungen nachweisen bzw. validieren will. Ist die Planung abgeschlossen setzt er den Staus von 'Nachweisplanung' zu 'Überprüfung der Nachweisplanung'. 
Ist die Planung in Ordnung wird die Validierung freigegeben und der Validierer kann die Nachweise für die Validierung selbst durchführen oder über Drittpersonen einholen. Sind alle notwendigen Nachweise vorhanden und damit die Nachweise vollständig, erfasst er einen Eintrag in der Rubrik 'Validierung der Anforderung (Durchführung)'. Ist dies erfolgt, kann der Validierer den Status von 'Freigegeben zur Validierung' zu 'Überprüfung der Validierung» weiterschalten (Kapitel xy). 
Ausserhalb der Bearbeitung von Anforderung hat der Validierer nur eine Leseberechtigung. Jeder Validierer kann jede Nachweisplanung und -durchführung bearbeiten. Dies ermöglicht eine grosse Flexibilität beim Wechsel des Validierers. Eine gleichzeitige Bearbeitung desselben Nachweises durch zwei Validierer wird verhindert. Der zweite Validierer wird darüber informiert, wer momentan den Nachweiseintrag bearbeitet.


% bishierher (3.1.3)






