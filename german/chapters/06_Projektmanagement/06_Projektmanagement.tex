
\clearpage
\section{Projektmanagement}

\vspace{\baselineskip}

Das Modul 'Projektmanagement' dient der Abnahme von Projekten. Sämtliche Mängel werden dokumentiert und mit Modulen (Arbeitspaketen) und Pendenzen verknüpft. Das Modul beinhaltet folgende 3 Kategorien:

\vspace{\baselineskip}

\begin{compactitem}
	\item Vorbehalte
	\item Pendenzen
	\item Module
\end{compactitem}

\vspace{\baselineskip}

\textbf{Vorbehalte:} Vorbehalten sind Mängel an Projektarbeiten, welche noch einer Korrektur oder Besserung bedürfen. Sie können mit Pendenzen oder Modulen verknüpft werden.

\vspace{\baselineskip}

\textbf{Pendenzen:} Die Pendenzen entsprechen den Pendenzen des Sitzungswesens und können einen Bestandteil eines Vorbehaltes sein. 

\vspace{\baselineskip}

\textbf{Module:} Sie beschreiben Arbeitsschritte oder -pakete eines Projekts. Die Vorbehalte beziehen sich jeweils auf eines dieser Module. Module erhalten einen Status, werlcher den Projektzustand wiedergibt (Beispielsweise: Freigegeben und Abgeschlossen). Das Statussystem kann kundenspezifisch konfiguriert werden.

\vspace{\baselineskip}

Mittels dem Modul 'Projektmanagement' haben Sie jederzeit den Überblick über eine Abnahme und wissen genau, welche Arbeiten noch erledigt oder nachgebessert werden müssen.

\vspace{\baselineskip}

Die detailierte Beschreibung dieser Funktionen erfolgt später. Bei Fragen wenden Sie sich an Ihren CUBE Support (E-Mail Adresse {\color{red} cube.support@emchberger.ch})


% \subsection{Vorbehalt}

% Text

% \subsection{Pendenzen}

% Text

% \subsection{Module}

% Text

% Dieses Kapitel folgt später.

