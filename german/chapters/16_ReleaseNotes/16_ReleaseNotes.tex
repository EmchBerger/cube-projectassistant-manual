
\clearpage
\section{Release Notes}

\textbf{Erläuterungen zu den Release Notes:}

\vspace{\baselineskip}

In diesem Kapitel finden Sie in Listenform die Änderungen im CUBE PA, welche jeweils in der neusten Version gegenüber der Vorgängerversion vorgenommen wurden. Dadurch erhalten Sie zusammengefasst einen Überblick über neue Funktionen und sonstigen Änderungen, welche im CUBE PA realisiert wurden.

\vspace{\baselineskip}

Bei Fragen bezüglich den Release Notes wenden Sie sich per Email an den CUBE PA Support: {\color{red} cube.support@emchberger.ch}

\vspace{\baselineskip}
% Ab hier die neuen RealeaseNotes einfügen %% jeweils die aktuellsten zu oberst

\textbf{Release Note Version 2.9 (Änderungen zu Version 2.8)} \\
\textbf{Dokumentenablage/Pläne:}
\begin{compactitem}
	\item Druckaufträge in 'Persönliche Projektübersicht' und 'Menü Dokumentenablage' 
	\item Lieferscheine von Druckaufträgen drucken und versenden
	\item Visualisierung Filter (alte Versionen, ungültige Versionen, Verbesserung der Lesbarkeit)
	\item Verschiedene Optimierungen in der Dokumentenablage
	\item Tag Struktur: neue Gliederung / Layout
	\item Änderungen bei Dokumentenberechtigungen: Neu 'untergeordnete Gruppen berechtigen'
\end{compactitem}
\textbf{Dossier:}
\begin{compactitem}
	\item Funktion Planungsdossier 
\end{compactitem}
\textbf{Kalender:}
\begin{compactitem}
	\item Auswahl aus den Dokumentenversionen bei den Links 
\end{compactitem}

\vspace{\baselineskip}

\textbf{Release Note Version 2.8 (Änderungen zu Version 2.7)} \\
\textbf{Dokumentenablage:}
\begin{compactitem}
	\item Neue Benutzerführung für das Tagging. Die gewünschten Tags können neu durch Browsen der gesamten Tag-Hierarchien gesetzt werden.
\end{compactitem}
\textbf{Sitzungswesen:}
\begin{compactitem}
	\item Sitzungsprotokolle können kopiert werden.
	\item Zu Pendenzen können individuell Beilagen abgelegt werden.
\end{compactitem}
\textbf{Adressliste:}
\begin{compactitem}
	\item Die Adressliste kann exportiert werden nach Excel.
\end{compactitem}
\textbf{Vorbehalte-/Mängelmodul:}
\begin{compactitem}
	\item Ein neues Modul zur Erfassung und Verfolgung von Vorbehalten und Mängeln wurde erstellt. Die Dokumentation wird nach Abschluss der Pilotphase im bestellenden Projekt nachgeführt.
\end{compactitem}
\textbf{Allgemein:}
\begin{compactitem}
	\item Diverse Sicherheits- und Stabilitätsverbesserungen.
	\item Fehlerkorrekturen
\end{compactitem}
\vspace{\baselineskip}

\textbf{Release Note Version 2.7 (Änderungen zu Version 2.6)} \\
\textbf{Sitzungswesen:}
\begin{compactitem}
  \item Als Beilagen zu Sitzungseinladungen, -protokollen sowie Pendenzen können neu bestehende, in der Dokumentenablage hinterlegte Dokumente verknüpft werden.
	\item Bei Sitzungen kann neu unterschieden werden zwischen Teilnehmern und Personen, die nur auf dem Protokollverteiler sind.
	\item Die Teilnehmerliste des Protokolls enthält neu die Funktion der Personen im Projekt.
	\item Standard-Teilnehmerlisten und Traktandenlisten können kopiert werden
	\item Sitzungseinladungen können direkt als .ics - Datei z.B. in den Outlook-Kalender importiert werden bzw. an alle Teilnehmer verschickt werden.
	\item Sitzungsprotokolle können direkt an die Teilnehmer + Verteilerliste als E-Mail verschickt werden, inkl. allen Beilagen.
\end{compactitem}
\textbf{Qualitätsmanagement, Handbücher:}
\begin{compactitem}
  \item Der gesamte Inhalt eines Handbuches inkl. Anschlussdokumente kann jetzt heruntergeladen werden.
\end{compactitem}
\textbf{Dokumentenablage, Planlieferung:}
\begin{compactitem}
  \item Die Lieferscheine / Mails z.Hd. der verschiedenen Empfänger können neu individuelle Mitteilungen enthalten.
\end{compactitem}
\textbf{System:}
\begin{compactitem}
  \item Die Systemplattform wurde von Symfony 2.8 auf Symfony 3.0 angehoben.
	\item Falls gültige Zugangsdaten (E-Mail Adresse und Passwort) im Browser hinterlegt sind, wird der Benutzer direkt eingeloggt.
\end{compactitem}

\vspace{\baselineskip}
\textbf{Release Note Version 2.6 (Änderungen zu Version 2.5)} \\
\begin{compactitem}
  \item Diverse Performance-Verbesserungen für Systeme mit sehr grossen Datenmengen.
\end{compactitem}

\vspace{\baselineskip}
\textbf{Release Note Version 2.5 (Änderungen zu Version 2.4)} \\
\textbf{Dokumentenablage:}
\begin{compactitem}
  \item Komplett überarbeitete Übersichtsseite. Die Filterkriterien werden fortlaufend übernommen.
	\item Filtereinstellungen (Tag-Navigation) werden beim Erfassen eines neuen Dokumentes übernommen.
	\item Upload von Dokumenten / Fotos direkt aus der mobilen App, inklusive Erfassung des GPS Stempels (sofern verfügbar). GPS Stempel werden auch beim Upload von Bildern durch nicht-mobile Geräte verarbeitet.
\end{compactitem}
\textbf{Sitzungswesen:}
\begin{compactitem}
  \item Berechtigung für Gruppen hinzugefügt.
	\item Automatischer Vorschlag für den Endzeitpunkt von Sitzungen.
\end{compactitem}

\vspace{\baselineskip}

\textbf{Release Note Version 2.4 (Änderungen zu Version 2.3)} \\
\textbf{Dokumentenablage:}
\begin{compactitem}
  \item Anstelle eines Dokumentes kann neu ein Link gespeichert werden.
	\item Überarbeitung des Layouts beim Dokumentenupload.
	\item Die Standardsortierung ist neu nach Titel (statt Zeitpunkt der Erstellung des Eintrages).
	\item Beim Setzen der Zugriffsberechtigungen werden für den ersten Eintrag automatisch alle Rechte (lesen, schreiben, löschen) angewählt. Für alle weiteren Einträge wird nur das Leserecht angewählt. Selbstverständlich können diese Einstellungen manuell angepasst werden.	
\end{compactitem}

\vspace{\baselineskip}

\begin{compactitem}
  \item \textbf{Sitzungswesen:} für künftige Sitzungen kann der Terminblocker direkt als .ics-Datei heruntergeladen werden.
	\item Import von Adressdaten aus Excel-Tabelle ist neu möglich.
	\item Verbessertes Login in den mobilen Apps.
	\item Single Sign-on: Nach dem Login in eine CUBE PA Instanz kann ohne erneutes Eingeben von Benutzername / Passwort direkt in alle anderen berechtigten Instanzen gewechselt werden. \textbf{Neu muss mit der E-Mail Adresse eingeloggt werden, der Benutzername wird nicht mehr verwendet.}
	\item Diverse Fehlerbehebungen (Projektplan wird nach Änderung der Filterkriterien nicht mehr korrekt visualisiert; Sortierung der Pendenzenliste bei Export in PDF nicht korrekt)	
\end{compactitem}

\vspace{\baselineskip}

\textbf{Release Note Version 2.3 (Änderungen zu Version 2.2)} \\
Der Fokus der Version 2.3 liegt auf der Dokumentenverwaltung sowie auf dem Einsatz von CUBE PA im Rahmen von Ausführungsprojekten im Bauwesen. Die wesentlichen Neuerungen sind:

\begin{compactitem}
	\item Einführung eines Dokumentenkorbes ähnlich des bekannten Warenkorbes aus Online-Shops. Über den Dokumentenkorb können verschiedene Funktionen wie der kombinierte Download als Zip-Archiv, Direktversand per Mail sowie Planrepro bedient werden.
	\item Direktversand von Mails aus CUBE PA mit angehängten Dokumenten. Der Versand von Dokumenten wird zu Gunsten der Nachvollziehbarkeit und Transparenz aufgezeichnet.
	\item Übermittlung von Plänen an Reprozentren zur Direktauslieferung von geplotteten Plänen. Mehrere Pläne können in Rahmen einer einzigen Bestellung an mehrere Empfänger verschickt werden, wobei die Ausführung (Anzahl Exemplare, Druck-/elektronische Kopie) pro Empfänger separat festgelegt werden kann.
	\item Hierarchisches Tagging: Dokumententags können neu in einer Art und Weise ähnlich der bekannten Ordnerstrukturen angelegt werden. Dies erlaubt die optimale, kombinierte Nutzung der Vorteile der beiden Ablagearten 'Tagging' und 'Ordner'. Dokumente können dabei sowohl über einzelne oder mehrere Tags wie auch hierarchisch durch Navigieren durch die gesamte Struktur gefunden werden. Dasselbe Dokument kann gleichzeitig in verschiedenen 'Hierarchiebäumen' auffindbar gemacht werden.
	
\end{compactitem}

\vspace{\baselineskip}

\textbf{Release Note Version 2.2 (Änderungen zu Version 2.1)} \\

\begin{compactitem}
	\item \textbf{Übersicht:} Neu werden in der Übersicht bei den Dokumenten die Optionen erst mit Klick auf den entsprechenden Dokumententitel geöffnet. Zudem kann in der Übersicht auch gerade die Vorschau eingeblendet werden. 
	\item \textbf{Benutzerverwaltung:} Neu enthalten einige Eingabefelder Hinweise zu ihrer Bedeutung und Funktionsweise.
	\item \textbf{Dokumentenablage:} Zugriffsrechte können neu gerade beim Erstellen eines Dokumenteneintrags vergeben werden.
	\item \textbf{iOS-App:} In der iOS-App gibt es neue Funktionen: Es ist möglich in einem Dokument zu zoomen, sowie Dokumente an Apps und Kontakte zu versenden
\end{compactitem}

\vspace{\baselineskip}

\textbf{Release Note Version 2.1 (Änderungen zu Version 2.0)} \\

\textbf{Dokumentenablage:}

\begin{compactitem}
	\item Beim Speichern von Dokumenten kann gewählt werden, ob es sich um eine Hauptversion oder Unterversion handelt.
	\item Ein Dokument kann neu mehreren Dossiers zugeordnet werden.
	\item Dossiers können neu gegen Veränderungen gesperrt werden. Bei gesperrten Dossiers können keine neuen Dokumente hinzugefügt oder entfernt werden.
	\item Über die Dossier-Ansicht ist neu die Dokumentenvorschau verfügbar und es besteht die Möglichkeit, Dokumente (versionsgenau) aus dem Dossier zu entfernen.
	\item Es gibt keine separate 'show'- und 'edit'-Ansicht mehr für die Dossiers. Das wurde vereint mit den Zugriffsrechten der Dokumentenablage; es wird jeweils das angezeigt, was über die Zugriffsrechte konfiguriert wurde.
	\item In der Dokumentenansicht ('show'-View) ist neu auch die Lokalisierungsvorschau enthalten.
	\item Neu wird beim Speichern des Dokumenteneintrages nur noch dann eine neue Version in der Datenbank erstellt, wenn auch ein neues Dokument hochgeladen wurde. Wenn nur die Metadaten angepasst wurden, wird dies zwar historisiert, aber ohne eine neue Version zu erstellen. Wenn ein Dokument ausgecheckt und ohne Änderung wieder eingecheckt wird oder wenn über das Drag+Drop-Feld das identische / unveränderte Dokument nochmals hochgeladen wird, wird ebenfalls keine neue Version erstellt.
	\item Metadaten von Dokumenten können auch für ältere Versionen verändert werden (z.B. nützlich um die Zugriffsrechte oder Tags / Dossiers zu verändern). Der Edit-Zugriff auf die älteren Versionen ist nur über die Änderungshistorie möglich.
	\item Ein Dokument kann mehreren Orten zugeordnet werden. 
\end{compactitem}

