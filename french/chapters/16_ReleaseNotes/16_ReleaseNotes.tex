
\clearpage
\section{Notes de version}

\textbf{Explications relatives aux notes de version :}

\vspace{\baselineskip}

Ce chapitre consiste en une liste des modification faites dans la dernière version de CUBE PA par rapport à la version précédente. Ceci apporte un aperçu des nouvelles fonctions et d'autres modifications faites dans CUBE PA.

\vspace{\baselineskip}

Pour toute question concernant les notes de version, veuillez adresser un e-mail au support CUBE : {\color{red} cube.support@emchberger.ch}

\vspace{\baselineskip}

 % Ab hier die neuen RealeaseNotes einfügen %% jeweils die aktuellsten zu oberst
\textbf{Note de version 2.8 (Modifications par rapport à la version 2.7)} \\
\textbf{Classement des documents :}
\begin{compactitem}
	\item Nouvelle manipulation pour l'attribution de tags. Les tags peuvent maintenant être sélectionnés en parcourant toute l'hiérrarchie de tags.
\end{compactitem}
\textbf{Gestion des séances :}
\begin{compactitem}
	\item Les procès-verbaux de séances peuvent être copiés.
	\item Des pièces jointes peuvent être ajoutées aux affaires en suspens individuelles.
\end{compactitem}
\textbf{Liste d'adresses :}
\begin{compactitem}
	\item La liste d'adresses peut maintenant être exportée en fichier Excel.
\end{compactitem}
\textbf{Module de défauts / réserves :}
\begin{compactitem}
	\item Nouveau module pour l'enregistrement et le suivi de défauts et de réserves. La documentation sera établie après l'achèvement de la phase pilote.
\end{compactitem}
\textbf{Système :}
\begin{compactitem}
	\item Diverses améliorations de sécurité et de stabilité.
	\item Correction d'erreurs.
\end{compactitem}
\vspace{\baselineskip}

\textbf{Note de version 2.7 (Modifications par rapport à la version 2.6)} \\
\textbf{Gestion des séances :}
\begin{compactitem}
  \item Des documents existants dans le classement des documents peuvent désormais être directement liés comme pièce jointe à une invitation à une séance, un procès-verbal de séance, et à une affaire en suspens.
	\item Nouvelle distinction entre participants à une séances et personnes sur la liste de distribution.
	\item La liste de participants dans le procès-verbal d'une séance contient désormais la fonction des participants dans le projet.
	\item Les listes de participants à une séance et les listes d'ordre du jour peuvent être copiées.
	\item Les invitations aux séances peuvent être directement importées en tant que fichiers .ics, dans les calendriers Outlook par exemple, et envoyées à tous les participants.
	\item Les procès-verbaux de séances peuvent être directement envoyés par e-mail aux participants sur la liste de distribution, y compris toutes leurs pièces jointes.
\end{compactitem}
\textbf{Gestion de la qualité, Manuels :}
\begin{compactitem}
  \item Le contenu entier d'un manuel, y.c. tout document additionnel, peut désormais être téléchargé.
\end{compactitem}
\textbf{Classement des documents, Livraison de plans :}
\begin{compactitem}
  \item Les bulletins de livraison / e-mails peuvent désormais être personnalisés pour chaque destinataire.
\end{compactitem}
\textbf{Système :}
\begin{compactitem}
  \item La plateforme du système a été améliorée de Symfony 2.8 à Symfony 3.0.
\end{compactitem}
 
\vspace{\baselineskip}
\textbf{Note de version 2.6 (Modifications par rapport à la version 2.5)} \\
\begin{compactitem}
  \item Diverses améliorations du rendement pour systèmes à vastes ensembles de données.
\end{compactitem}

\vspace{\baselineskip}

\textbf{Note de version 2.5 (Modifications par rapport à la version 2.4)} \\
\textbf{Classement des documents :}
\begin{compactitem}
  \item Page index complètement révisée. Les paramètres de filtre sont directement appliqués.
	\item Les paramètres de filtre (navigation des tags) sont appliqués lors de la création d'un nouveau document.
	\item Le chargement de documents / photos (y.c. marque GPS si disponible) est désormais possible directement depuis l'application mobile. Les marques GPS sont aussi traitées lors du chargement de photos depuis des appareils non mobiles.
\end{compactitem}
\textbf{Gestion des séances :}
\begin{compactitem}
  \item Droit d'accès pour groupes ajoutés.
	\item Suggestion automatique de l'heure de fin d'une séance.
\end{compactitem}

\vspace{\baselineskip}


\textbf{Note de version 2.4 (Modifications par rapport à la version 2.3)} \\

\textbf{Classement des documents :}
\begin{compactitem}
	\item Au lieu d'un fichier, un lien peut désormais être enregistré.
	\item Modification de la mise en page lors du chargement d'un document.
	\item L'ordre des documents par défaut se fait maintenant selon le titre (au lieu du temps de création de la saisie).
	\item Lors de la sélection des droits d'accès, tous les droits (lire, modifier, supprimer) sont automatiquement attribués pour la première saisie. Pour toute saisie supplémentaire, le droit de lecture seulement est fixé. Ces sélections peuvent ensuite être manuellement modifiées.
\end{compactitem}
\textbf{Gestion de séances :}
\begin{compactitem}
	\item Pour les séances futures, le rendez-vous peut être directement téléchargé sous forme de fichier .ics.
\end{compactitem}
\textbf{Divers :}
\begin{compactitem}
	\item L'importation d'adresses d'un tableau Excel est maintenant possible.
	\item Login amélioré dans les applications mobiles.
	\item Login unique : Après un login dans une instance CUBE PA, l'utilisateur peut directement passer à toute autre instance à laquelle il est autorisé sans devoir effectuer un nouveau login. \textbf{Le login s'effectue désormais avec l'adresse e-mail. Le nom d'utilisateur ne sera plus utilisé.}
	\item Nombreuses corrections et améliorations.
\end{compactitem}

\vspace{\baselineskip}

\textbf{Note de version 2.3 (Modifications par rapport à la version 2.2)} \\

Les nouveautés de la version 2.3 sont concentrées sur le classement des documents ainsi que sur l'utilisation de CUBE PA dans le cadre de projets d'exécution dans le domaine de la construction. Les nouveautés essentielles sont :
\begin{compactitem}
	\item Introduction d'un panier de documents similaire au panier d'achat d'un online-shop. Le panier de documents apporte plusieurs fonctionnalités comme le téléchargement de plusieurs documents en tant qu'archive Zip, leur envoi direct par e-mail, ainsi que leur envoi pour impression.
	\item Envoi direct d'e-mails depuis CUBE PA avec documents annexés. L'envoi de documents est enregistré dans l'historique des documents pour assurer la transparence et la traçabilité.
	\item Transmission de plans aux centres d'impression pour livraison directe de plans imprimés. Plusieurs plans peuvent être envoyés à différents destinataires moyennant une seule commande ; l'exécution se fait séparément pour chaque destinataire (nombre d'exemplaires, version imprimées/électroniques).
	\item Structure de tags hiérarchique : Les tags de documents peuvent maintenant être utilisés de manière similaire à la classification par dossiers. Ceci permet une utilisation optimale combinée des avantages des deux types de classifications 'tags' et 'dossiers'. Les documents peuvent être recherchés par le moyen d'un ou plusieurs tags aussi bien qu'en navigant la structure hiérarchique de classement de documents. Un même document peut être classé et retrouvé dans plusieurs 'arbres hiérarchiques'.
\end{compactitem}

\vspace{\baselineskip}

\textbf{Note de version 2.2 (Modifications par rapport à la version 2.1)} \\

\begin{compactitem}
	\item \textbf{Aperçu :} Dans l'aperçu des documents, cliquer sur le titre d'un document dévoile maintenant les options. De plus, la prévisualisation d'un document peut maintenant se faire directement dans l'aperçu.
	\item \textbf{Gestion des utilisateurs :} Des indications relatives aux fonctions des différents champs ont été ajoutées.
	\item \textbf{Classement des documents :} Les droits d'accès peuvent maintenant être directement définis lors de la création d'une saisie de document.
	\item \textbf{Application iOS :} Nouvelles fonctions dans l'application iOS : Il est maintenant possible d'agrandir un document, ainsi que d'envoyer des documents à des applications et des contacts.
\end{compactitem}

\vspace{\baselineskip}

\textbf{Note de version 2.1 (Modifications par rapport à la version 2.0)} \\

\textbf{Classement des documents :}

\begin{compactitem}
	\item En sauvegardant un document, l'utilisateur peut choisir s'il s'agit d'une version principale ou secondaire du document.
	\item Un document peut désormais être classé dans plusieurs dossiers.
	\item Les dossiers peuvent désormais être verrouillés contre les modifications. Pour les dossiers verrouillés, il n'est pas possible d'ajouter de nouveaux documents ni d'enlever des documents existants.
	\item La visualisation d'un dossier supporte une prévisualisation des documents contenus dans le dossier et permet d'enlever des documents du dossier (versions spécifiques).
	\item Il n'y a plus de modes 'visualisation' et 'modification' séparés pour les dossiers. Ceci a été combiné avec les droits d'accès au classement des documents. La configuration des droits d'accès est montrée.
	\item La prévisualisation du lieu est désormais disponible dans le mode 'visualisation' d'un document.
	\item Lors de la sauvegarde d'un document, une nouvelle version est créée dans la banque de données uniquement si un nouveau document a été chargé. Si seules les métadonnées ont été modifiées, ceci est enregistré sans créer une nouvelle version du document. Si un document est extrait puis réintroduit sans modifications, ou si un document identique est rechargé dans la zone de glisser-déposer, une nouvelle version du document ne sera pas créée.
	\item Les métadonnées d'un document peuvent aussi être modifiées pour des anciennes versions du document (ceci est utile par exemple pour changer les droits d'accès ou les tags / dossiers). L'accès modification de versions anciennes du document est uniquement accessible via l'historique.
	\item Un document peut être lié à plusieurs lieux.
\end{compactitem}

