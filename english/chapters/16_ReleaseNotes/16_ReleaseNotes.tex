
\clearpage
\section{Release Notes}

\textbf{Release notes explanations:}

\vspace{\baselineskip}

This chapter consists of a list of changes which were made in the latest CUBE PA version in comparison with the previous version. This gives an overview of new functions and other changes made in CUBE PA.

\vspace{\baselineskip}

In case of questions about the release notes, please contact the CUBE PA support: {\color{red} cube.support@emchberger.ch}

\vspace{\baselineskip}

 % Ab hier die neuen RealeaseNotes einfügen %% jeweils die aktuellsten zu oberst
\textbf{Release note 2.8 (Changes to version 2.7)} \\
\textbf{Document management:}
\begin{compactitem}
	\item New handling for tag attribution. Tags can now be selected by browsing the complete tag hierarchy.
\end{compactitem}
\textbf{Meeting management:}
\begin{compactitem}
	\item Meeting minutes can now be copied.
	\item Attachments can now be added to individual actions.
\end{compactitem}
\textbf{Address list:}
\begin{compactitem}
	\item The address list can now be exported as an Excel file.
\end{compactitem}
\textbf{Reservations module:}
\begin{compactitem}
	\item New module for the logging and follow-up of reservations. The documentation will be issued after completion of the pilot phase.
\end{compactitem}
\textbf{System :}
\begin{compactitem}
	\item Diverse security and stability improvements.
	\item Bug fixes.
\end{compactitem}
\vspace{\baselineskip}

\textbf{Release note 2.7 (Changes to version 2.6)} \\
\textbf{Meeting management:}
\begin{compactitem}
  \item For meeting invitations and meeting minutes as well as actions, existing documents under document management can now be linked to as attachments.
	\item It can now be differentiated between participants and persons who are only on the distribution list.
	\item The participants list in meeting minutes now includes the function of the participants in the project.
	\item Les standard meeting participant and agenda items lists can now be copied.
	\item Meeting invitations can now be directly imported as .ics file, in Outlook calenders for example, and sent to all participants.
	\item Meeting minutes can directly be sent via e-mail to the participants on the distribution list, including all their attachments.
\end{compactitem}
\textbf{Quality management, Handbooks:}
\begin{compactitem}
  \item The entire content of a handbook, including any additional documents can now be downloaded.
\end{compactitem}
\textbf{Document management, Plan deliveries:}
\begin{compactitem}
  \item The delivery slips / e-mails to recipients can now be personalized for each recipient.
\end{compactitem}
\textbf{System:}
\begin{compactitem}
  \item The system platform was upgraded from Symfony 2.8 to Symfony 3.0.
\end{compactitem}
 
\vspace{\baselineskip}
\textbf{Release note 2.6 (Changes to version 2.5)} \\
\begin{compactitem}
  \item Various performance improvements for systems with large datasets.
\end{compactitem}

\vspace{\baselineskip}

\textbf{Release note 2.5 (Changes to version 2.4)} \\
\textbf{Document management:}
\begin{compactitem}
  \item Completely revised index page. Filter criteria are automatically applied.
	\item Filter settings (tag navigation) are applied when creating a new document.
	\item Uploading documents / photos (incl. GPS stamps if available) directly from the mobile app is now possible. GPS stamps are also processed when uploading photos from non-mobile devices.
\end{compactitem}
\textbf{Meeting management:}
\begin{compactitem}
  \item Access rights for groups added.
	\item Automatic suggestion for end-time of a meeting.
\end{compactitem}

\vspace{\baselineskip}

\textbf{Release note 2.4 (Changes to version 2.3)} \\

\textbf{Document management:}
\begin{compactitem}
	\item A link can now be saved instead of a file.
	\item Layout changes when uploading a document.
	\item Documents are now sorted by default according to document title (instead of entry creation date).
	\item When selecting access rights, all rights (read, write, delete) are automatically assigned to the first entry. For each additional entry, only the reading right is assigned. These settings can be changed manually.
\end{compactitem}
\textbf{Meeting management}
\begin{compactitem}
	\item For future meetings, the appointments can be directly downloaded as .ics files.
\end{compactitem}
\textbf{Various:}
\begin{compactitem}
	\item Importing addresses from an Excel sheet is now possible.
	\item Improved login from mobile applications.
	\item Unique login: After logging in to a CUBE PA instance, the user can switch to any other instance for which they are authorized without having to login again. \textbf{The login is now done with the e-mail address. The username will no longer be used.}
	\item Various bug fixes.
\end{compactitem}

\vspace{\baselineskip}

\textbf{Release note 2.3 (Changes to version 2.2)} \\

The focus of version 2.3 is on document management as well as on using CUBE PA for execution projects in the construction field. The essential new functionalities are:
\begin{compactitem}
	\item Introduction of a document cart similar to a shopping cart in an online shop. The document cart offers many functionalities such as downloading several documents as a ZIP archive, sending them directly by e-mail, and sending them for printing.
	\item Directly sending e-mails with document attachments from CUBE PA. Sending documents is saved in the document log for transparency and traceability.
	\item Transmission of plans to plot centers for delivery of printed plans. Several plans can be sent to different recipients with one order. The execution is carried out separately for each recipient (number of copies, printed/electronic versions).
	\item Hierarchic tag structure: Document tags can now be used similarly to folder classification. This allows an optimal combined usage of the advantages of both classification types 'tags' and 'folders'. Documents can be found through one or more tags as well as by navigating the hierarchical document management structure. The same document can be classified and found in several 'hierarchic trees'.
\end{compactitem}

\vspace{\baselineskip}

\textbf{Release note 2.2 (Changes to version 2.1)} \\

\begin{compactitem}
	\item \textbf{Overview:} In the document overview, clicking on a document title now unveils the options. In addition, document previews are now directly available in the overview.
	\item \textbf{User management:} Hints to the functions of the different fields were added.
	\item \textbf{Document management:} Access rights can now directly be attributed when creating a document.
	\item \textbf{iOS application:} New functionality in the iOS application: It is now possible to zoom in on a document, as well as to send documents to applications and contacts.
\end{compactitem}

\vspace{\baselineskip}

\textbf{Release note version 2.1 (Changes to version 2.0)} \\

Document management:

\begin{compactitem}
	\item When saving a document, the user can choose whether the document is a major or a minor version.
	\item A document can now be assigned to several dossiers.
	\item Dossiers can now be locked against changes. No new documents can be added to and no documents can be removed from locked dossiers.
	\item The dossier view enables the preview of documents and removing documents from the dossier (specific versions).
	\item There are no more separate 'show' and 'edit' views for the dossiers. This has been combined with access rights to document management. The configuration over the access rights is shown.
	\item In the document view ('show' view), the location preview is now available.
	\item When saving a document entry, a new version is now created in the database only if a new document has been uploaded. If only the metadata has been modified, this is chronicled without creating a new document version. If a document is checked out then checked back in without any changes, or if an identical document is re-uploaded in the drag and drop field, no new document version will be created.
	\item Document metadata can also be modified for older document versions (e.g. useful to change the access rights or the tags / dossiers). The edit access of older document versions is only accessible via the document change log.
	\item A document can be assigned to multiple locations. 
\end{compactitem}

